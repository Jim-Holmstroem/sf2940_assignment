\documentclass[a4paper,twoside=false,abstract=false,numbers=noenddot,
titlepage=false,headings=small,parskip=half,version=last]{scrartcl}
\usepackage{lib/header}
\usepackage{lib/probability}
\begin{document}
\generateheader{Assignment 1}

\begin{exercise}{1: 1.12.3.10} {\bf Intersection of sigma algebras}   \\
    $\FF_1$ and $\FF_2$ are two sigma algebras of subsets of $\Omega$. Show
    that
    \begin{equation*}
        \FF_1 \cap \FF_2
    \end{equation*}
    is a sigma algebra of subsets of $\Omega$.
\end{exercise}
\begin{solution}
\end{solution}
\pagebreak

\begin{exercise}{2: 2.6.5.6} {\bf Use Chen's Lemma} \\
    $X \in \distr{Po}{\lambda}$. Show that
    \begin{equation}
        \Expected[X^n] =
        \lambda \sum\limits_{i=0}^{n-1} \binom{n-1}{k}\Expected[X^k].
    \end{equation}
    Aid: Use Chen's Lemma with suitable $H(x)$.
\end{exercise}
\begin{solution}
    \begin{lemma}
        \label{lemma:chen}  % TODO verify the correctness
        {\bf Chen's Lemma} $X \in \distr{Po}{\lambda}$ and $H(x)$ is a bounded
        Borel function, then
        \begin{equation*}
            \Expected[XH(X)] = \lambda\Expected[H(X+1)].
        \end{equation*}
    \end{lemma}
\end{solution}
\pagebreak

\begin{exercise}{3: 3.8.3.1}
    {\bf Joint Distributions \& Conditional Expectations } \\
    Let $(X, Y)$ is a bivariate random variable, where $X$ is discrete and $Y$
    is continuous. $(X, Y)$ has a joint probability mass - and density function
    given by
    \begin{equation*}
        f_{X,Y}(k, y) = \begin{cases}
            \partdev{P(X=k, Y\le y)}{y} =
                \lambda\frac{(\lambda y)^k}{k!}e^{-2\lambda y}
                    &,\, k\in\ZZ_{\ge 0},\, y\in [0, \infty) \\
            0       &.
        \end{cases}
    \end{equation*}
    (a) Check that
    \begin{equation*}
        test
    \end{equation*}
\end{exercise}
\begin{solution}
\end{solution}
\pagebreak

\begin{exercise}{4: 3.8.3.14} Title of the problem \\
\end{exercise}
\begin{solution}
\end{solution}
\pagebreak

\begin{exercise}{5: 4.7.2.4} Title of the problem \\
\end{exercise}
\begin{solution}
\end{solution}
\pagebreak

\begin{exercise}{6: 5.8.3.11} Title of the problem \\
\end{exercise}
\begin{solution}
\end{solution}
\pagebreak

\begin{exercise}{7: 7.6.1.1} Title of the problem \\
\end{exercise}
\begin{solution}
\end{solution}
\pagebreak

%-----------------------
\end{document}
